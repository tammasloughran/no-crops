% Draft manuscript on forestation sensitivity experiments.
% Note: Figures are plotted using .png files for now since SVG rendering is very
% slow for LaTeX. At a later date I will change the figures to .svg files. For example:
%\includesvg[width=\linewidth]{../plots/cVeg_aus_anom.svg}

% Preamble.
\documentclass[]{article}
\usepackage[a4paper, total={170mm,257mm}, left=20mm, top=20mm,]{geometry}
\usepackage{svg} % Use SVGs only for final draft.
\usepackage{float} % Allows the [H] option meaning "place figure here and only here"
\usepackage[section]{placeins} % Defines \FloatBarrier so that figures can't move.
\usepackage{subcaption} % Allows subfigures and captions
\usepackage{lineno} % Enable line numbers.
\usepackage{textcomp}
\usepackage[style=authoryear]{biblatex}
\usepackage{amsmath}
\addbibresource{references.bib}
\setlength{\parindent}{1cm}
\linenumbers

\title{Sensitivity experiments of Australian forestation in ACCESS-ESM1.5 under several global warming levels}
\author{Tammas Loughran et al.}

\begin{document}

\maketitle

\begin{center}
    \Large
    \vspace{0.9cm}
    \textbf{Abstract}
\end{center}

[Placeholder]

\raggedright
\parindent=.35in % Setting raggedright removes paragraph indents. This puts them back.

\section{Introduction}

Meeting the net-zero commitments of the Paris Agreement will require not only fossil fuel mitigation but also active carbon dioxide removal (CDR) strategies since some sources of greenhouse gasses are difficult to abate immediately (citations).
Australia's long-term emissions reduction strategy relies on a large portion of its emissions reduction coming from land-based solutions to generate negative emissions, such as by forests and soil carbon (citations).
Emerging technologies also may leverage biomass as a fuel source, which when combined with carbon capture and storage allows for permanent removal of CO$_2$ (citations).
The preservation for natural forests and development of managed forests is therefore a key resource to achieve net-zero by 2050, and its land-use requirements compete with agricultural requirements, particularly in an expanding population and economy (citations).
The land-use changes required would potentially involve some abandonment of marginal traditional agricultural lands to make way for forests, either for commercial harvesting or as new protected natural lands.
The growth of new forests can affect the climate in various ways: by carbon assimilation with contributed to the reduction of atmospheric CO$_2$ concentrations (c), by altering physical surface properties which in turn may either warm or cool surface temperatures and alter the hydrological cycle (c), and finally by emission of reactive volatile organic compounds \parencite{weber_chemistry_albedo_2024}.

%What are the types of forestation land use change, afforestation reforrestation

% Findings of existing studies eg loughran et al.

The CO$_2$ removal potential and climate mitigation of forestation is still poorly understood for Australia.
Integrated assessment models in the past have provided plausible scenarios of global forestation for CMIP6, but these scenarios do not have any additional forest area expansion in Australia \parencite{loughran_limited_2023} and are therefore unsuitable for investigating the CDR potential of Australian forests and the biogeophysical impacts.
Furthermore, we do not have projections under stable climates at different global warming levels that are consistent with the Paris agreement targets.
% This is probably better as a separate paper.
%In fact, how ACCESS responds to forestation in general is still poorly understood and fundamental land-use change experiments are needed to establish the suitability of the model for assessing the CDR potential of Australia's terrestrial biosphere.
Therefore, we aim to create a range of projections of the CDR potential of forestation for both Australia and the globe under a variety of possible stabilized future climate states.

We run ACCESS-ESM1-5 under pre-industrial and present day and future climate to estimate the productivity and climate effects of forests under different mitigation scenarios.
We also create forestation scenarios with specific biome types to differentiate the climate effects from planting large forests of specific plant functional types.

\section{Methods}

\subsection{Model description}

We use the coupled Earth system model ACCESS-ESM1-5 to simulate the climate for scenarios with global-scale forest expansion.
ACCESS-ESM1-5 is a fully coupled atmosphere, land, ocean and cryosphere model that includes a fully interactive and dynamic carbon cycle.
It is composed of the UM atmosphere in the XXX configuration (c), the MOMX ocean model (c) with WOMBAT ocean biogeochemistry (c), CICE sea ice model () and CABLE3 for the land surface model (c).
The land surface vegetation is represented as sub-grid fractional tiles.
Figure \ref{fig:dominant_pfts} demonstrates the dominant vegetation type or land surface type at each grid-cell.
The global distribution of vegetation, crops and land surface types is determined by ...
% nutrient limitations included in CABLE


\begin{figure}[H]
    \begin{subfigure}[b]{\linewidth}
        \centering
        \includegraphics[width=\linewidth]{../plots/cable_dominant_tiles.png}
    \end{subfigure}
    \caption{CABLE dominant PFTs}
    \label{fig:dominant_pfts}
\end{figure}

\subsection{Experiments}

To project forestation scenarios that are relevant to the targets of Paris Agreement, we utilize the existing Global Warming Level (GWL) experiments conducted with ACCESS-ESM1-5 first proposed by \cite{king_studying_2021} as reference simulations, upon which forestation land-use changes are applied. The GWL simulations branch from the high emissions scenario \textit{ssp-585} (c) at regular 5-year intervals over the period 2030–2060. We apply forestation after 400 years of simulation to ensure the climate has reached a reasonable stabilization. At that point the GWL simulations represent stabilized climates ranging 1.5–3 \textcelsius warming relative to pre-industrial conditions. On long timescales, there still exists some climate drift driven by the slow processes of the oceanic circulation (known as zero-emissions commitment (ZEC); c Chamberlain et al.), however, for the purposes of examining the impact of forestation on climate (on time-scales of 200 years), the amount of drift is negligible.

We apply forestation as a replacement of croplands with any of three forest vegetation types.
In CABLE, there are four vegetation types that represent forests: evergreen broad leaf, evergreen needle leaf, deciduous broad leaf and deciduous needle leaf.
Deciduous needle leaf forests were excluded from forestation because they are limited to high latitudes of the Asian continent (Fig. \ref{fig:dominant_pfts}), and very few grid cells have both deciduous needle leaf forests and nearby croplands on which to expand.
The resulting forest expansion for complete replacement of crops with forests is shown in Fig. (\ref{fig:forestation_on_crops}).
A large amount of forestation occurs in India, where a large portion of existing croplands currently exists.
Of particular note is Australia, which features forest expansion in two distinct regions: the southeast and the southwest.

The procedure for replacing crops with trees is summarized in Fig. \ref{fig:to_forest}.
For each grid-cell, crops are removed by reducing the tile fraction for croplands.
The forest biome/vegetation type fractions are increased by the same amount, while maintaining the relative proportions of the tree types in that grid cell.
If, for a given grid-cell there are crops but no trees, then trees are allocated to the crop fractions using the proportions of the nearest forested grid-cell, ensuring that the forest expansion is appropriate for the local climate.
If any of the resulting tree fractions are below a negligible threshold, then that grid-cell has crops replaced with the dominant tree type from the nearest grid-cell that has trees (this is due to the choice of technical implementation of vegetation types in CABLE3).

We apply forestation instantaneously to croplands on the basis that if the climate is suitable to grow crops for a given region, then is feasible that forests can be grown there also. In reality, there would be competition between the economical demands to expand agriculture and the lands suitable for forestry and natural forest expansion.
Since our main focus is on the effects of forestation on climate and not, for example, crop yields, we ignore the potential economic drawbacks of the occupation of lands by forests.
The potential for forests to coexist with agriculture is left for future studies.
%It's possible that marginal grasslands/rangelands could be used for forestation (c )

Table \ref{tab:experiments} shows a summary of all of the forestation experiments. We conduct three basic types of experiments:

\begin{itemize}
    \item Forestation under GWLs
    \item Forestation using a single vegetation type
    \item Partial deployment
\end{itemize}

% The experiment naming conventions used here are inconsistent and are subject to change to something that is easier to understand for this paper.
The global warming level simulations (2030–2060) test the effects of forestation under different climate states.
The forestation using a single vegetation type tests the effects of forestation assuming that forestation is implemented with a single species.
While this may not be realistic for some regions (for example planting evergreen broad leaf forests in mid–high latitudes) it simplifies examining regional climate effects of particular vegetation types, rather than a mixture of types that occurs in the standard GWL forestation simulations.
The partial deployment experiments feature forestation on 50\%, 25\% or 10\% of all croplands per grid-cell.
These experiments aim to represent a sample of more practical/achievable deployment of forestation as a CDR strategy, with a view to determining the how much forestation would be needed to have a significant impact on climate.

Fig. \ref{fig:experiment_branching} shows a diagram of the branching points for the GLW forestation experiments, the partial deployment and single vegetation types experiments.
The partial deployment and single vegetation type experiments branch from the 2030 GWL.
All experiments are compared relative to their respective GWL experiment from which they are branched.

Fig. \ref{fig:forestation_ammount} shows the global area extent of existing forests in ACCESS-ESM1-5 and the extent of additional forestation occuring in each experiment.
In general, existing forest area (dark green) decreases over the course of the \textit{ssp-585} scenario and croplands increases.
The amount of forestation occurring in the GWL experiments is ~19 M km$^2$, this is an expansion of ~39\% of global forest area.
The 10\% deployment scenario forest expansion is comparable to that of the LUMIP forestation scenario in \cite{loughran_limited_2023}.
% maybe also show the forestation extent for Australia if I'm also going to have a section just for Australia.
In Australia there is 0.58 M km$^2$ expansion of forest area, or a 61\% increase in Australia's forest area.

\begin{figure}[H]
    \begin{subfigure}[b]{\linewidth}
        \centering
        \includegraphics[width=0.5\linewidth]{../plots/crop_to_forest1.png}
    \end{subfigure}
    \begin{subfigure}[b]{\linewidth}
        \centering
        \includegraphics[width=0.5\linewidth]{../plots/nearest_forest.png}
    \end{subfigure}
    \begin{subfigure}[b]{\linewidth}
        \centering
        \includegraphics[width=0.5\linewidth]{../plots/minimum.png}
    \end{subfigure}
    \caption{Replace crops}
    \label{fig:to_forest}
\end{figure}

\begin{table}[]
    \caption{List of experiments and reference simulations.}
    \label{tab:experiments}
    \begin{tabular}{llll}
\hline
Exp name             & Branches from & Forestation on               & Compared to       \\ \hline
esm-esm-piNoCrops    & piControl     & All crops at 1850            & esm-piControl     \\
esm-esm-piNoCrops-02 & piControl     & All crops at 1850            & esm-piControl     \\
GWL-EGNL-B2030       & PI-GWL-t6     & Evergreen needle leaf on all crops at 2030            & PI-GWL-t6         \\
GWL-EGBL-B2030       & PI-GWL-t6     & Evergreen broad leaf on all crops at 2030            & PI-GWL-t6         \\
GWL-DCBL-B2030       & PI-GWL-t6     & Deciduous broad leaf on all crops at 2030            & PI-GWL-t6         \\
GWL-50pc-B2030       & PI-GWL-t6     & 50\% of 2030 croplands       & PI-GWL-t6         \\
GWL-25pc-B2030       & PI-GWL-t6     & 25\% of 2030 croplands       & PI-GWL-t6         \\
GWL-10pc-B2030       & PI-GWL-t6     & 10\% of 2030 croplands       & PI-GWL-t6         \\
GWL-NoCrops-B2030    & PI-GWL-t6     & All crops at 2030            & PI-GWL-t6         \\
GWL-NoCr-B2030-02    & PI-GWL-t6     & All crops at 2030            & PI-GWL-t6         \\
GWL-NoCrops-B2035    & PI-GWL-B2035  & All crops at 2035            & PI-GWL-B2035      \\
GWL-NoCrops-B2040    & PI-GWL-B2040  & All crops at 2040            & PI-GWL-B2040      \\
GWL-NoCrops-B2045    & PI-GWL-B2045  & All crops at 2045            & PI-GWL-B2045      \\
GWL-NoCrops-B2050    & PI-GWL-B2050  & All crops at 2050            & PI-GWL-B2050      \\
GWL-NoCrops-B2055    & PI-GWL-B2055  & All crops at 2055            & PI-GWL-B2055      \\
GWL-NoCrops-B2060    & Pi-GWL-B2060  & All crops at 2060            & Pi-GWL-B2060      \\
GWL-NoCr-B2060-02    & Pi-GWL-B2060  & All crops at 2060            & Pi-GWL-B2060      \\
GWL-EqFor-B2060      & Pi-GWL-B2060  & Only crops at 2030           & Pi-GWL-B2060      \\
GWL-CO2only-B2030    & PI-GWL-t6     & No land-use change           & GWL-NoCrops-B2030 \\ \hline
\end{tabular}
\end{table}

\begin{figure}[H]
    \centering
    \begin{subfigure}[b]{\linewidth}
        \includegraphics[width=\linewidth]{../plots/tree_area_diff.png}
    \end{subfigure}
    \caption{Regions of forestation in the no-crop experiments.}
    \label{fig:forestation_on_crops}
\end{figure}

\begin{figure}[H]
    \centering
    \begin{subfigure}[b]{\linewidth}
        \includegraphics[width=\linewidth]{../plots/experiment_branching.png}
    \end{subfigure}
    \caption{Diagram of experiment branching points.}
    \label{fig:experiment_branching}
\end{figure}

\begin{figure}[H]
    \centering
    \begin{subfigure}[b]{\linewidth}
        \includegraphics[width=\linewidth]{../plots/forestation_sensitivity_amount.png}
    \end{subfigure}
    \caption{Area extent of total global existing forests in each simulation (dark green) and the additional forestation replacing crop lands (light green).}
    \label{fig:forestation_ammount}
\end{figure}

\section{Results}

\subsection{Carbon cycle}

Fig. \ref{fig:global_co2} shows the global mean atmospheric CO$_2$ concentration in the GWL and corresponding forestation experiments.
The atmospheric CO$_2$ decreases by 30 ppm in the 2030 GWL and by up to 40 ppm in the 2060 GWL.
The light blue line show the experiment for the 2060 GWL using the same area of forest expansion that occurs in the 2030 GWL. This experiment has approximately the same CO$_2$ concentration as the 2060 GWL forestation, which indicates that the difference in forest extent between the 2030 GWL and 2060 GWL is negligible.
The warmer and higher the atmospheric CO$_2$ concentration, the greater the CO$_2$ removal, likely because warmer conditions are more favorable to vegetation growth and the CO$_2$ fertilization effect.
This is supported by Fig. \ref{fig:global_cLand}, which shows the difference of total land carbon from forestation relative to the respective GWL simulation.
150–180 Pg(C) is taken up by the land surface by forestation, depending on the climate state and atmospheric CO$_2$ concentration.

% Might cut this. only mildly interesting.
Fig. \ref{fig:global_cpools} shows global mean time series of each carbon pool.
~85\% is contained in wood, and is therefore available to be used as wood products if forestation is implemented as wood production plantations.
CABLE's soil carbon pools show decreases in response to increases in forest cover.
This is due to a transition to slower decay timescales in the woody forest vegetation types which features a higher proportion of lignin.
This loss of soil carbon slowly recovers over time, but does not fully recover on the time scale of 200 years. 

Fig. \ref{fig:map_cLand} shows the 30-year mean at the end of simulation for the 2050 GWL (for example).
The distribution of land carbon uptake mirrors the expansion of forests cover as demonstrated in Fig. \ref{fig:forestation_on_crops}.
There are a few small regions that undergo decreases in total land carbon, particularly in the alpine region of South Asia.

The uptake of carbon by Australia is shown in Fig. \ref{fig:australia_cLand}.
Approximately 5–12 Pg(C) is taken up by Australia after 100 years of forest establishment ($\sim$1 Gt(C)/year).
There is less distinction between the different climate states when looking at the Australian region, due to the greater regional impacts internal climate variability.
However, the highest GWL ensemble members (red lines) is in general larger than the lowest GWL ensemble members (pink lines).
Hence, the effects of CO$_2$ fertilization are also evident for the Australian region alone.

\begin{figure}[H]
    \centering
    \begin{subfigure}[b]{\linewidth}
        \includesvg[width=\linewidth]{../plots/co2_global_warming_level_no_crops.svg}
    \end{subfigure}
    \caption{Global mean surface CO$_2$ concentrations for each experiment. Thin lines are the standard global warming level simulations. Bold lines are the forestation experiments.}
    \label{fig:global_co2}
\end{figure}

\begin{figure}[H]
    \centering
    \begin{subfigure}[b]{\linewidth}
        \includegraphics[width=\linewidth]{../plots/cLand_GWL_gloabl_sum.png}
    \end{subfigure}
    \caption{Global sum of the total land carbon content difference between the global warming level simulations and the forestation experiments.}
    \label{fig:global_cLand}
\end{figure}

\begin{figure}[H]
    \centering
    \begin{subfigure}[b]{\linewidth}
        \includegraphics[width=\linewidth]{../plots/cpools_gloabl_sum.png}
    \end{subfigure}
    \caption{Global sum of the carbon pools' difference between the global warming level simulations and the forestation experiments.}
    \label{fig:global_cpools}
\end{figure}

\begin{figure}[H]
    \centering
    \begin{subfigure}[b]{\linewidth}
        \includegraphics[width=\linewidth]{../plots/cLand_GWL-NoCrops-B2050_last20.png}
    \end{subfigure}
    \caption{Global map of 2070--2100 mean total land carbon content difference between the global warming level simulations and the forestation experiments 2050 branching point.}
    \label{fig:map_cLand}
\end{figure}

%\begin{figure}[H]
%    \centering
%    \begin{subfigure}[b]{\linewidth}
%        \includegraphics[width=\linewidth]{../plots/cLand_GWL-NoCrops-B2050_australia_last20.png}
%    \end{subfigure}
%    \caption{Australian map of 2070--2100 mean total land carbon content difference between the global warming level simulations and the forestation experiments 2050 branching point.}
%    \label{fig:aus_map_cLand}
%\end{figure}

\begin{figure}[H]
    \centering
    \begin{subfigure}[b]{\linewidth}
        \includegraphics[width=\linewidth]{../plots/cLand_GWL_australia_sum.png}
    \end{subfigure}
    \caption{Australian sum of the total land carbon content difference between the global warming level simulations and the forestation experiments.}
    \label{fig:australia_cLand}
\end{figure}

\subsection{Climate}

\subsubsection{Temperature}

Fig. \ref{fig:global_temperature_timeseries} shows the global mean surface air temperature decrease from forestation under each global warming level.
The greatest cooling occurs in the highest global warming level (2060) of -0.4 \textcelsius, and the least cooling occurs in the lowest global warming level (2030) by -0.2 \textcelsius{}.
Global temperatures have approximately stabilized by 200 years of simulation.

%\begin{figure}[H]
%    \centering
%    \begin{subfigure}[b]{\linewidth}
%        \includegraphics[width=\linewidth]{../plots/tas_timeseries.png}
%    \end{subfigure}
%    \caption{Global mean surface temperature for the piControl simulation and the piNoCrops simulation.}
%    \label{fig:pre-industrial_temperature_tseries}
%\end{figure}
%
%\begin{figure}[H]
%    \centering
%    \begin{subfigure}[b]{\linewidth}
%        \includegraphics[width=\linewidth]{../plots/temperature_diff.png}
%    \end{subfigure}
%    \caption{Surface temperature difference between the piControl simulation and the piNoCrops simulation, averaged over the last 20 years of each simulation.}
%    \label{fig:pre-industrial_temperature_map}
%\end{figure}

\begin{figure}[H]
    \centering
    \begin{subfigure}[b]{\linewidth}
        \includegraphics[width=\linewidth]{../plots/tas_GWL_gloabl_mean.png}
    \end{subfigure}
    \caption{Timeseries of the change in global mean surface air temperature from forestation, expressed as a difference between the forestation experiment and the standard global warming level simulation.}
    \label{fig:global_temperature_timeseries}
\end{figure}

\begin{figure}[H]
    \centering
    \begin{subfigure}[b]{\linewidth}
        \includegraphics[width=\linewidth]{../plots/tas_GWL-NoCrops-B2060_last30.png}
    \end{subfigure}
    \caption{Map of the change in surface air temperature from forestation averaged over the last 30
    years of simulation, expressed as a difference between the forestation experiment and the standard global warming level simulation.}
    \label{fig:temperature_map}
\end{figure}

\subsubsection{Precipitation}

[Placeholder]

\section{Discussion}

Limitations of using ACCESS: no fire, no tree death or demography, no pft-climate dynamics, under estimation of evapotranspiration/hydrology leading to dominance of albedo, there are decrease in soil carbon with forestation and soil carbon uncertainties.

\section{Conclusion}

[Placeholder]

\printbibliography

\section{Supplementary Material}
\setcounter{figure}{0}

\end{document}
