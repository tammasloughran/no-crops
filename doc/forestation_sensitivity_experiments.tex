% Draft manuscript on forestation sensitivity experiments.
% Note: Figures are plotted using .png files for now since SVG rendering is very
% slow for LaTeX. At a later date I will change the figures to .svg files. For example:
%\includesvg[width=\linewidth]{../plots/cVeg_aus_anom.svg}

% Preamble.
\documentclass[]{article}
\usepackage[a4paper, total={170mm,257mm}, left=20mm, top=20mm,]{geometry}
\usepackage{svg} % Use SVGs only for final draft.
\usepackage{float} % Allows the [H] option meaning "place figure here and only here"
\usepackage[section]{placeins} % Defines /FloatBarrier so that figures can't move.
\usepackage{subcaption} % Allows subfigures and captions
\usepackage{lineno} % Enable line numbers.
\usepackage{textcomp}
\usepackage[style=authoryear]{biblatex}
\usepackage{amsmath}
%\addbibresource{forestation_sensitivity_experiments.bib}
\setlength{\parindent}{1cm}
\linenumbers

\title{Sensitivity experiments of Australia forestation in ACCESS-ESM1.5 under several global warming levels}
\author{Tammas Loughran et al.}

\begin{document}

\maketitle

\begin{center}
    \Large
    \vspace{0.9cm}
    \textbf{Abstract}
\end{center}

[Placeholder]

\raggedright
\parindent=.35in % Setting raggedright removes paragraph indents. This puts them back.

\section{Introduction}

The CO$_2$ removal potential and climate mitigation of forestation is still poorly understood for Australia.
Integrated assessment models produced a plausible scenario of global forestation for CMIP6 but these scenarios do not have any additional forest area expansion in Australia and are therefore unsuitable for answering the question.

We need additional simulations to answer these questions.
Furthermore, we do not have projections under stanble climates at different global warming levels that are consistent with the Paris agreement targets.
In fact, how ACCESS responds to forestation in general is still poorly understood and fundamental land-use change experiments are needed.

\section{Methods}

Description of ACCESS. where are dominant PFTs
Description of crop removal and replacement with trees

\subsection{Experiments}

\begin{itemize}
\item piControl
\item piControl-no-crops
\item GWL branching points
\item GWL branching points without crops
\end{itemize}


\begin{figure}[H]
    \centering
    \begin{subfigure}[b]{\linewidth}
        \includegraphics[width=\linewidth]{../plots/crop_to_forest1.png}
    \end{subfigure}
    \caption{replace crops}
    \label{fig:to_forest}
\end{figure}
\begin{figure}[H]
    \centering
    \begin{subfigure}[b]{\linewidth}
        \includegraphics[width=\linewidth]{../plots/nearest_forest.png}
    \end{subfigure}
    \caption{replace crops}
    \label{fig:nearest_forest}
\end{figure}
\begin{figure}[H]
    \centering
    \begin{subfigure}[b]{\linewidth}
        \includegraphics[width=\linewidth]{../plots/minimum.png}
    \end{subfigure}
    \caption{replace crops}
    \label{fig:minimum}
\end{figure}

\subsection{Statistical methods}

[Placeholder]

\section{Results}

[Placeholder]

\subsection{Carbon cycle}

[Placeholder]

\begin{figure}[H]
    \centering
    \begin{subfigure}[b]{\linewidth}
        \includesvg[width=\linewidth]{../plots/co2_global_warming_level_no_crops.svg}
    \end{subfigure}
    \caption{replace crops}
    \label{fig:global_co2}
\end{figure}

\subsection{Climate}

[Placeholder]

\section{Discussion}

\subsection{Soil carbon decrease}

\section{Conclusion}

[Placeholder]

%\printbibliography

\section{Supplementary Material}
\setcounter{figure}{0}


\end{document}

